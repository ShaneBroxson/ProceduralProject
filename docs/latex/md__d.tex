One paragraph of project description goes here. Briefly address the 5 W\textquotesingle{}s. What is the project? Who made it? Did you make it independently or with a team? When did you make it? At what level of experience and proficiency were / are you? Why did you make it? For fun? For an assignment? How does it relect you as a person? Where did you make it? What school / class? Note for school project\+: this may be the only thing someone visiting your page reads. Try to make a great impresison. Make someone want to hire you. This section will also be useful to you in the future to remind you about the project. \subsection*{Demonstration}

Add an animated gif of your program running. You can use ShareX, G\+I\+P\+HY Capture or another tool. \href{https://blog.github.com/2018-06-29-GIF-that-keeps-on-GIFing/}{\texttt{ https\+://blog.\+github.\+com/2018-\/06-\/29-\/\+G\+I\+F-\/that-\/keeps-\/on-\/\+G\+I\+Fing/}} Note for school project\+: This is a great way for friends and family to easily see your project in action. The markdown is  \subsection*{Documentation}

Create a docs folder in your project. If using Java, generate Java\+Doc in your I\+DE. If using C++, use Doxygen. Change repository settings (using Settings at top of page) to use Git\+Hub Pages with your docs folder. Add a link to the javadoc/doxygen index.\+html file with this markdown\+: \mbox{[}Text to appear\mbox{]}(U\+RL) The U\+RL will be Your\+Git\+Hub\+User\+Name.\+github.\+io/\+Your\+Repository\+Name/foldername/filename Do not include the docs folder name in your U\+RL. Sample\+: \href{https://pv-cop.github.io/PV-README-TEMPLATE/javadoc/index.html}{\texttt{ Java\+Doc}} \subsection*{Diagrams}

\subsection*{Getting Started}

Instructions to get a copy of the project up and running on someone\textquotesingle{}s local machine for development and testing purposes. Note for real project\+: You want to be able to share your project and enable collaboration. Note for school project\+: You want a potential client or hirer to be able to run your program. \subsection*{Built With}


\begin{DoxyItemize}
\item C\+Lion by Jet\+Brains
\item Visual Studio Compiler
\item Min\+GW Compiler \subsection*{Contributing}
\end{DoxyItemize}

G\+UI Development is most likely the next development process for this project. \subsection*{Author}


\begin{DoxyItemize}
\item Shane Broxson \subsection*{License}
\end{DoxyItemize}

No current license. \subsection*{Acknowledgments}


\begin{DoxyItemize}
\item Stack\+Over\+Flow
\item cplusplus.\+com
\item Prof. Scott Vanselow \subsection*{History}
\end{DoxyItemize}

Started with all vectors being passed in seperatly to deal and save all data being worked on. Added files to save data on. Added code to populate vectors with existing data when program starts. Changed vectors to be added to struct at beginning of file and associated function calls. \subsection*{Key Programming Concepts Utilized}